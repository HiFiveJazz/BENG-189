\documentclass{report}

\input{preamble}
\input{macros}
\input{letterfonts}
\usepackage{pgfplots}
\usepackage{xcolor}
\usepackage{amsmath}
\usepackage{graphicx}
\usepackage{tikz}
\tikzstyle{startstop} = [rectangle, rounded corners, minimum width=3cm, minimum height=1cm, text centered, draw=black, fill=red!30]
\tikzstyle{process} = [rectangle, minimum width=3cm, minimum height=1cm, text centered, draw=black, fill=orange!30]
\tikzstyle{decision} = [diamond, minimum width=3cm, minimum height=1cm, text centered, draw=black, fill=green!30]
\tikzstyle{arrow} = [thick,->,>=stealth]
\title{\Huge{BENG 189}\\University of California San Diego}
\author{\huge{Jasmeet Bhatia}}
\date{}
\begin{document}

\maketitle
\newpage% or \cleardoublepage
% \pdfbookmark[<level>]{<title>}{<dest>}
\pdfbookmark[section]{\contentsname}{toc}
\tableofcontents
\pagebreak

\chapter{Lecture 1: Renal System}
\section{Renal System Part 1:}
\begin{figure}[htbp]
  \centering
  \includegraphics[width=0.4\textwidth]{img1.png}
  \caption{Anatomy of Kidneys, showing efferent and afferent blood vessesl}
\end{figure}
The kidneys sit below. They have two artery branches that feed oxygenated blood to the kidneys. Deoxygenated blood returns back to the \underline{Inferior Vena Cava}. The kidneys filter nutrition which passes through the \underline{ureter} which leads to the bladder. The kidneys sit below the ribs
\begin{figure}[htbp]
  \centering
  \includegraphics[width=0.4\textwidth]{img2.png}
  \caption{Specific anatomy of the kidney. This lecture focused primarily on the Medulla and Bowman's Capsule, discussed later this lecture}
\end{figure}
\begin{figure}[htbp]
  \centering
  \includegraphics[width=0.4\textwidth]{img3.png}
  \caption{Blood Flow of Urinary Tract}
\end{figure}
\section{Steps of Blood Flow}
\begin{itemize}
  \item Initial: Blood arrives via the areriole \item 1. Kidneys dump solutes and water in to the urinary tract
  \item 2. A delicate dance occurs between reabsorbing things and letting things being released to the urine
    \begin{itemize}
      \item Ex: Retain items such as glucose and amino acids
    \end{itemize}
  \item 3. Excrete some things
    \begin{itemize}
      \item Ex: \underline{Drugs, urea}, etc
    \end{itemize}
  \item 4. Items go towards the \underline{Bladder}
\end{itemize}
\dfn{Urinary Excretion}{
Excretion = Filtration - Reabsorption + Secretion
}
\section{How Different Solutes are Treated Differently}
\begin{figure}[htbp]
  \centering
  \includegraphics[width=0.4\textwidth]{img4.png}
  \caption{Flow of Solutes through the urinary system. The loopy of Henley is key to allowing the concentration of solutes}
\end{figure}
\begin{verbatim}
Glucose and Amino Acids are reabsorbed right away at 100% performance. 
Optimal Balance $\rightarrow$ excreting "bad" substances effeciently may \underline{require, a 
little loss of water.}
Dump blood flow into the Bowman's capsule
The textbook starts at the Glomerulus (G), travels to the Afferent arteriole (AA).
Leaky capillaries lose some water and small molecules. 
Proteins are too large to leave and thus they stay. This creates an osmotic pressure against 
filtering out water.
blood pressure, can lead to multi-organ failure
\end{verbatim}
\begin{figure}[htbp]
  \centering
  \includegraphics[width=0.4\textwidth]{img5.png}
  \caption{There exists $10^6$ nephrons, connected in parallel}
\end{figure}
  \item Proximal Tubulue
  \item this is your plasma without proteins
  \item Na reabsorption $\rightarrow$ water goes too
  \item This functions to decrease the volume of urine
  \item Loop of Henle $\rightarrow$ Na concentration changes
  \item Ascending Limb $\rightarrow$ Metabolically Active, Impermeable to $H_2O$, Permeable to $Na^+$  
  \item $Na^+$ is pumped from tubule $\rightarrow$ intersititium $\rightarrow$ leads to concentration gradient $\rightarrow$ Descending Loop of Henle
  \item Descending Loop of Henle $\rightarrow$ proggressively increasing [$Na^+$]
  \item Peritubule capillaries $\rightarrow$ pick up/ return fluid to interstitial vein
  \begin{figure}[htbp]
    \centering
    \includegraphics[width=0.7\textwidth]{img7.png}
    \caption{Concentration of Urine}
  \end{figure}
  Distil tube collect duct are \underline{imperm} to $H_2O$
  Distil tube collect duct are  \underline{perm} to $H_2O$ 
  fluid at the of LH of it is excreted large vlume dilute -> low concentration
  water leave as it descends smaller concentretated volume

  \ex{Diseases}{

  }

  \section{Dynamics of $Na^+$ and $H_2O$ Transport}
  \begin{figure}[htbp]
    \centering
    \includegraphics[width=0.4\textwidth]{img6.png}
    \caption{}
  \end{figure}
  \dfn{Flow Rate}{
  let Q(x) be the flower rate past x
  Assume in steady state
  let f(x) = the flow through the walls per unit length at x
  \begin{center}
  \\ \therefore Q(a) = Q(b) + \int_{a}^{b}{f$H_2O$(x)\,dx }
  \\ \therefore 0 = \frac{\partial Q(b)}{\partial x} + $f_{H_2O}(b)$
  \\ \therefore 0 = \frac{\partial Q}{\partial x} + $f_{H_2O}(x)$
  \\ Now consider the concentration of $Na^+$ and its concentration, c(x)
  \\ The amount of $Na^+$ transport along the tubule by flow as Q(x) \dot c(x) 
  \\ While the flow through the walls, $f_{Na^+}$
  \\ \[ Q(a)c(a) = Q(b)c(b) + \int_{a}^{b}{f_{Na^+}\,dx}
  \\ \therefore 0 = \frac{\partial}{\partial x} (Q(b)c(b)) + f_{Na^+}(b)
  \\ \therefore 0 = \frac{\partial}{\partial x} (Q(c)) + f_{Na^+}(x)
  \\ Q(a)c(a) = Q(b)c(b) + \int_{a}^{b}{f_{Na^+}\,dx} \]
  \end{center}
  }
  \goal{Model of Loop of Henle}{
  Describe a salty meal versus drinking water 
  }
\dfn{Shannon's Definition of Information (h)}{Let an ensemble, X, have input ($x, A_x, P_x$) where x is a random variable, $A_x$ is a set of possible outcomes ($a_1,a_2,a_3, ... a_1$) and where $P_x$ is the set of possibilities $P_x = (P_1,P_2, ... P_i)$ such that P(x=$a_i$) = $P_i$, we define Shannon's information content as the following: \[ x = a_i h(x=a_i) = log_2 \] 
\[ \therefore h(x = a_i) = log_2 \frac{1}{P(x=a_i)}  \]}

\end{document}
